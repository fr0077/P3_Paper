
\section{まとめ}
\begin{itemize}
\item[$\clubsuit$]当初の目的である中性子のスピン状態の重ね合わせによる干渉を観測することが出来た。\\
\item[$\clubsuit$]振動数$B$:\text{干渉において最重要なパラメータであり、高い精度で理論の予測と実験結果が一致した。}\\
\item[$\clubsuit$]位相$C$:\text{共鳴からのずれを考慮することによって定量的に説明出来た。}\\
\item[$\clubsuit$]オフセット$D$:共鳴からのずれ、バックグラウンドを考慮すれば、理論と実験結果のずれは定性的に理解が可能である。\\
\item[$\clubsuit$]振幅$A$:\text{共鳴からのずれ、バックグラウンドを考慮すれば、理論と実験結果のずれは理解が可能な範囲である。}\\
しかし、依然として以下の課題及び考慮していない事柄が存在する。
\end{itemize}
\paragraph{課題}
$\clubsuit$
オフセット$D$、振幅$A$の理論と実験結果のずれを定量的に説明すること。\\
\paragraph{考慮していない事柄}
\hspace{-9.5pt}1.中性子のスピンが空気によって反転すること\\
2.不均一磁場によって中性子が力を受けること\\

