
\section{まとめ}
\begin{itemize}
\item[$\clubsuit$]当初の目的である中性子のスピン状態の重ね合わせによる干渉を観測することが出来た。\\
\item[$\clubsuit$]実験結果は振幅と位相のずれは理論による予測と一致しなかったが、振動数及びオフセットは高い精度で一致し非常に満足のいく結果となった。\\
\item[$\clubsuit$]実験で得られた干渉パターンは様々な効果を考慮することによってその大部分は定性的に理解できた。\\
\\
しかし、依然として以下の課題及び考慮していない事柄が存在する。
\end{itemize}
\subsubsection{課題}
$\clubsuit$
電流の増加に伴い振幅が減少することの理論的説明。\\
但し、振幅が有意に減少しているかは今回の実験では判断出来ず、誤差の範囲で説明が可能かもしれない。
いずれにせよ、より詳細な実験が必要である。
\subsubsection{考慮していない事柄}
\hspace{-9.5pt}1.空気が中性子のスピンに及ぼす影響\\
2.中性子が磁場中で受ける力の影響\\
3.磁場が一様でないことの影響\\
4.中性子生成の際に発生する$\gamma$線が中性子の検出に及ぼす影響\\
5.\mbox{高速中性子のTOF分布が熱中性子の領域までテールを引いてることの影響}\\
