
\section{まとめ}
\begin{itemize}
\item[$\clubsuit$]当初の目的である中性子のスピン状態の重ね合わせによる干渉を観測することが出来た。\\
\item[$\clubsuit$]振幅A、位相C、オフセットDは理論による予測と一致しなかった。\\
\item[$\clubsuit$]干渉において最重要なパラメータである振動数Bは高い精度で一致し非常に満足のいく結果となった。\\
\item[$\clubsuit$]様々な効果を考慮することによって位相Cの理論と実験結果のずれを定量的に説明出来た。\\
しかし、依然として以下の課題及び考慮していない事柄が存在する。
\end{itemize}
\subsubsection{課題}
$\clubsuit$
振幅A、オフセットDの理論と実験結果のずれを定量的に説明すること。\\
\subsubsection{考慮していない事柄}
\hspace{-9.5pt}1.空気が中性子のスピンに及ぼす影響\\
2.中性子が磁場中で受ける力の影響\\
3.磁場が一様でないことの影響\\
4.中性子生成の際に発生する$\gamma$線が中性子の検出に及ぼす影響\\
5.\mbox{高速中性子のTOF分布が熱中性子の領域までテールを引いてることの影響}\\
