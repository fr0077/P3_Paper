\def\vector#1{\mbox{\boldmath $#1$}}
領域Ⅱにおける磁場は
\begin{align}
\vector{B}=B_{r}\cos(w_{s}t)\vector{\hat{x}}+B_{r}sin(w_{s}t)\vector{\hat{y}}+B_{z}\vector{\hat{z}}
\end{align}
とする。
\begin{align}
{\hbar}w_{r}=|{\mu}B_{r}|
\end{align}
\begin{align}
{\hbar}w_{z}=|{\mu}B_{z}|
\end{align}
とする。この時、領域Ⅱにおけるシュレディンガー方程式は
\begin{align}
i{\hbar}\frac{\partial {\psi}_{Ⅱ}(x,t)}{\partial t}=\left(-\frac{\hbar^2}{2m}\frac{\partial^2}{\partial x^2}+{\hbar}w_{r}\cos(w_{s}t){\sigma}_{x}+{\hbar}w_{r}\sin(w_{s}t){\sigma}_{y}+{\hbar}w_{z}{\sigma}_{z}\right){\psi}_{Ⅱ}(x,t)
\end{align}
と書ける。ここで、
\begin{align}
{\hbar}w_{r}\cos(w_{s}t){\sigma}_{x}+{\hbar}w_{r}\sin(w_{s}t){\sigma}_{y}+{\hbar}w_{z}{\sigma}_{z}=
\begin{pmatrix}
{\hbar}w_{z} &{\hbar}w_{r}e^{-iw_{s}t} \\
{\hbar}w_{r}e^{iw_{s}t} &-{\hbar}w_{z}
\end{pmatrix}
\end{align}
$と書ける。角速度w_{s}でz軸の周りに回転するユニタリー変換$
\begin{align}
U_{T}=\exp(iw_{s}t{\sigma}_{z})=
\begin{pmatrix}
e^{iw_{s}t} &0 \\
0 &e^{-iw_{s}t}
\end{pmatrix}
\end{align}
をもちいて
\begin{align}
U_{T}\begin{pmatrix}
{\hbar}w_{z} &{\hbar}w_{r}e^{-iw_{s}t} \\
{\hbar}w_{r}e^{iw_{s}t} &-{\hbar}w_{z}
\end{pmatrix}U_{T}^{\dagger}=
\begin{pmatrix}
{\hbar}w_{z} &{\hbar}w_{r} \\
{\hbar}w_{r} &-{\hbar}w_{z}
\end{pmatrix}
\end{align}
が成り立つ。
\begin{align}
{\psi}_{R}(x,t)=U_{T}{\psi}(x,t)
\end{align}
$とすると{\psi}_{R}(x,t)の満たすシュレディンガー方程式は$
\begin{align}
i{\hbar}\frac{\partial {\psi}_{R}(x,t)}{\partial t}=\left(-\frac{\hbar^2}{2m}\frac{\partial^2}{\partial x^2}+{\hbar}w_{r}{\sigma}_{x}+({\hbar}w_{z}-\frac{1}{2}{\hbar}w_{s}){\sigma}_{z}\right){\psi}_{R}(x,t)
\end{align}
となる。いま、共鳴条件
\begin{align}
{\epsilon}=\frac{1}{2}{\hbar}w_{s}-{\hbar}w_{z}=0
\end{align}
が成り立っているとする。この時、シュレディンガー方程式は
\begin{align}
i{\hbar}\frac{\partial {\psi}_{R}(x,t)}{\partial t}=\left(-\frac{\hbar^2}{2m}\frac{\partial^2}{\partial x^2}+{\hbar}w_{r}{\sigma}_{x}\right){\psi}_{R}(x,t)
\end{align}
いま、
\begin{align}
U_{D}=\exp(i{\pi}{\sigma}_{y}/4)=
\begin{pmatrix}
1/\sqrt{2} &1/\sqrt{2} \\
-1/\sqrt{2} &1/\sqrt{2}
\end{pmatrix}
\end{align}
というユニタリー変換を用いると、
\begin{align}
U_{D}{\sigma}_{x}U_{D}^{\dagger}={\sigma}_{z}
\end{align}
が成り立つ。
\begin{align}
{\psi}_{D}(x,t)=U_{D}{\psi}_{R}(x,t)
\end{align}
$とすると、{\psi}_{D}(x,t)の満たすシュレーディンガー方程式は$
\begin{align}
i{\hbar}\frac{\partial {\psi}_{D}(x,t)}{\partial t}=\left(-\frac{\hbar^2}{2m}\frac{\partial^2}{\partial x^2}+{\hbar}w_{r}{\sigma}_{z}\right){\psi}_{D}(x,t)
\end{align}
$この方程式の解のうち、エネルギー固有値がE_{n}={\hbar}w_{n}であるものは$
\begin{align}
{\psi}_{D}(x,t) =
\begin{pmatrix}
A_{n}^{+}e^{ik_{n}^{+}x}+B_{n}^{+}e^{-ik_{n}^{+}x}  \\
A_{n}^{-}e^{ik_{n}^{-}x}+B_{n}^{-}e^{-ik_{n}^{-}x}
\end{pmatrix}
e^{-iw_{n}t}
\end{align}
ここで、
\begin{align}
\frac{{{\hbar}^2}{k_{n}^{\pm}}^2}{2m}{\pm}{\hbar}w_{r}=E_{n}
\end{align}
が成り立つ。ゆえに、領域Ⅱにおける波動関数は
\begin{align}
{\psi}_{Ⅱ}(x,t)=U_{T}^{\dagger}U_{D}^{\dagger}{\psi}_{D}(x,t)=\frac{1}{\sqrt{2}}
\begin{pmatrix}
\left[(A_{n}^{+}e^{ik_{n}^{+}x}+B_{n}^{+}e^{-ik_{n}^{+}x})+(A_{n}^{-}e^{ik_{n}^{-}x}+B_{n}^{-}e^{-ik_{n}^{-}x})\right]e^{-i(w_{n}+w_{s}/2)t} \\
\left[(A_{n}^{+}e^{ik_{n}^{+}x}+B_{n}^{+}e^{-ik_{n}^{+}x})-(A_{n}^{-}e^{ik_{n}^{-}x}+B_{n}^{-}e^{-ik_{n}^{-}x})\right]e^{-i(w_{n}-w_{s}/2)t}
\end{pmatrix}
\end{align}
いま、入射波、すなわち領域Ⅰにおける波動関数を
\begin{align}
{\psi}_{Ⅰ}(x,t)=
\begin{pmatrix}
e^{-iw_{z}x/v} \\
0
\end{pmatrix}
e^{ik_{0}x}e^{-iw_{0}t}
\end{align}
$とする。ここで、v=\frac{{\hbar}k_{0}}{m}であり、$
\begin{align}
\frac{{{\hbar}^2}k_{0}^2}{2m}=E_{0}={\hbar}w_{0}
\end{align}
$が成り立つ。x=0における接続条件により、以下の4つの式が任意のtについて成り立つ。$
\begin{align}
e^{-iw_{0}t}=\left[(A_{n}^{+}+B_{n}^{+})+(A_{n}^{-}+B_{n}^{-})\right]e^{-i(w_{n}+w_{s}/2)t}
\end{align}
\begin{align}
0=\left[(A_{n}^{+}+B_{n}^{+})-(A_{n}^{-}+B_{n}^{-})\right]e^{-i(w_{n}-w_{s}/2)t}
\end{align}
\begin{align}
(k_{0}-w_{z}/v)e^{-iw_{0}t}=\left[k_{n}^{+}(A_{n}^{+}-B_{n}^{+})+k_{n}^{-}(A_{n}^{-}-B_{n}^{-})\right]e^{-i(w_{n}+w_{s}/2)t}
\end{align}
\begin{align}
0=\left[k_{n}^{+}(A_{n}^{+}-B_{n}^{+})-k_{n}^{-}(A_{n}^{-}-B_{n}^{-})\right]e^{-i(w_{n}-w_{s}/2)t}
\end{align}
これにより、
\begin{align}
w_{n}=w_{0}-w_{s}/2
\end{align}
$を満たすn以外のnについて$
\begin{align}
A_{n}^{+}=B_{n}^{+}=A_{n}^{-}=B_{n}^{-}=0
\end{align}
となる。また、
\begin{align}
w_{n}=w_{0}-w_{s}/2
\end{align}
$を満たすnを1とすると、A_{1}^{\pm},B_{1}^{\pm}は0ではない。以上から、$
\begin{align}
{\psi}_{Ⅱ}(x,t)=&\frac{1}{\sqrt{2}}
\begin{pmatrix}
\left[(A_{1}^{+}e^{ik_{1}^{+}x}+B_{1}^{+}e^{-ik_{1}^{+}x})+(A_{1}^{-}e^{ik_{1}^{-}x}+B_{1}^{-}e^{-ik_{1}^{-}x})\right]e^{-i(w_{1}+w_{s}/2)t} \\
\left[(A_{1}^{+}e^{ik_{1}^{+}x}+B_{1}^{+}e^{-ik_{1}^{+}x})-(A_{1}^{-}e^{ik_{1}^{-}x}+B_{1}^{-}e^{-ik_{1}^{-}x})\right]e^{-i(w_{1}-w_{s}/2)t}
\end{pmatrix}\\
=&\begin{pmatrix}
\left[(A_{1}^{+}e^{ik_{1}^{+}x}+B_{1}^{+}e^{-ik_{1}^{+}x})+(A_{1}^{-}e^{ik_{1}^{-}x}+B_{1}^{-}e^{-ik_{1}^{-}x})\right]e^{-iw_{0}t} \\
\left[(A_{1}^{+}e^{ik_{1}^{+}x}+B_{1}^{+}e^{-ik_{1}^{+}x})-(A_{1}^{-}e^{ik_{1}^{-}x}+B_{1}^{-}e^{-ik_{1}^{-}x})\right]e^{-i(w_{0}-w_{s})t}
\end{pmatrix}
\end{align}
$透過波、すなわち領域Ⅲにおける波動関数のうち、エネルギー固有値がE_{n}であるものは以下のようなものである。$
\begin{align}
{\psi}_{Ⅲ}(x,t)=
\begin{pmatrix}
C_{n}^{+}e^{iK_{n}^{+}x} \\
C_{n}^{-}e^{iK_{n}^{-}x}
\end{pmatrix}
e^{-iw_{n}t}
\end{align}
ここで
\begin{align}
\frac{{{\hbar}^2}{K_{n}^{\pm}}^2}{2m}{\pm}{\hbar}w_{z}={\hbar}w_{n}
\end{align}
$が成り立つ。x=dにおける接続より、以下の4つの式が任意のtについて成り立つ。$
\begin{align}
C_{n}^{+}e^{iK_{n}^{+}d}e^{-iw_{n}t}=\left[(A_{1}^{+}e^{ik_{1}^{+}d}+B_{1}^{+}e^{-ik_{1}^{+}d})+(A_{1}^{-}e^{ik_{1}^{-}d}+B_{1}^{-}e^{-ik_{1}^{-}d})\right]e^{-iw_{0}t}
\end{align}
\begin{align}
C_{n}^{-}e^{iK_{n}^{-}d}e^{-iw_{n}t}=\left[(A_{1}^{+}e^{ik_{1}^{+}d}+B_{1}^{+}e^{-ik_{1}^{+}d})-(A_{1}^{-}e^{ik_{1}^{-}d}+B_{1}^{-}e^{-ik_{1}^{-}d})\right]e^{-i(w_{0}-w_{s})t}
\end{align}
\begin{align}
iK_{n}^{+}C_{n}^{+}e^{iK_{n}^{+}d}e^{-iw_{n}t}=\left[ik_{1}^{+}(A_{1}^{+}e^{ik_{1}^{+}d}-B_{1}^{+}e^{-ik_{1}^{+}d})+ik_{1}^{-}(A_{1}^{-}e^{ik_{1}^{-}d}-B_{1}^{-}e^{-ik_{1}^{-}d})\right]e^{-iw_{0}t}
\end{align}
\begin{align}
iK_{n}^{-}C_{n}^{-}e^{iK_{n}^{-}d}e^{-iw_{n}t}=\left[ik_{1}^{+}(A_{1}^{+}e^{ik_{1}^{+}d}-B_{1}^{+}e^{-ik_{1}^{+}d})-ik_{1}^{-}(A_{1}^{-}e^{ik_{1}^{-}d}-B_{1}^{-}e^{-ik_{1}^{-}d})\right]e^{-i(w_{0}-w_{s})t}
\end{align}
$いま、w_{n}=w_{0}でもw_{n}=w_{0}-w_{s}でもないnについては$
\begin{align}
C_{n}^{+}=C_{n}^{-}=0
\end{align}
が成り立つ。ゆえに、
\begin{align}
w_{2}=w_{0}-w_{s}
\end{align}
$とすると、残るのはnが0と2だけで、さらに$
\begin{align}
C_{0}^{-}=C_{2}^{+}=0
\end{align}
透過波はこれらを重ね合わせた
\begin{align}
{\psi}_{Ⅲ}(x,t)=
\begin{pmatrix}
C_{0}^{+}e^{iK_{0}^{+}x} \\
0
\end{pmatrix}
e^{-iw_{0}t}+
\begin{pmatrix}
0 \\
C_{2}^{-}e^{iK_{2}^{-}x}
\end{pmatrix}
e^{-iw_{2}t}
=\begin{pmatrix}
C_{0}^{+}e^{iK_{0}^{+}x}e^{-iw_{0}t} \\
C_{2}^{-}e^{iK_{2}^{-}x}e^{-i(w_{0}-w_{s})t}
\end{pmatrix}
\end{align}
である。ここで、
\begin{align}
\frac{{{\hbar}^2}{K_{n}^{\pm}}^2}{2m}{\pm}{\hbar}w_{z}={\hbar}w_{n}
\end{align}
により、
\begin{align}
\frac{{{\hbar}^2}{K_{0}^{+}}^2}{2m}+{\hbar}w_{z}={\hbar}w_{0}
\end{align}
\begin{align}
\frac{{{\hbar}^2}{K_{2}^{-}}^2}{2m}-{\hbar}w_{z}={\hbar}(w_{0}-w_{s})
\end{align}
よって
\begin{align}
K_{0}^{+}=\sqrt{\frac{2m}{{\hbar}^2}({\hbar}w_{0}-{\hbar}w_{z})}{\simeq}k_{0}-w_{z}/v
\end{align}
\begin{align}
K_{2}^{-}=\sqrt{\frac{2m}{{\hbar}^2}({\hbar}(w_{0}-w_{s})+{\hbar}w_{z})}{\simeq}k_{0}+(w_{z}-w_{s})/v
\end{align}
よって
\begin{align}
{\psi}_{Ⅲ}(x,t)=
\begin{pmatrix}
C_{0}^{+}e^{iK_{0}^{+}x}e^{-iw_{0}t} \\
C_{2}^{-}e^{iK_{2}^{-}x}e^{-i(w_{0}-w_{s})t}
\end{pmatrix}
{\simeq}
\begin{pmatrix}
C_{0}^{+}e^{i(k_{0}-w_{z}/v)x}e^{-iw_{0}t} \\
C_{2}^{-}e^{i(k_{0}+(w_{z}-w_{s})/v)x}e^{-i(w_{0}-w_{s})t}
\end{pmatrix}=
\begin{pmatrix}
C_{0}^{+}e^{i(-w_{z}/v)x} \\
C_{2}^{-}e^{i((w_{z}-w_{s})/v)x}e^{iw_{s}t}
\end{pmatrix}
e^{ik_{0}x}e^{-iw_{0}t}
\end{align}
ここで、共鳴条件
\begin{align}
w_{s}=2w_{z}
\end{align}
により、
\begin{align}
{\psi}_{Ⅲ}(x,t)=&
\begin{pmatrix}
C_{0}^{+}e^{i(-w_{z}/v)x} \\
C_{2}^{-}e^{i(-w_{z}/v)x}e^{iw_{s}t}
\end{pmatrix}
e^{ik_{0}x}e^{-iw_{0}t}\\
=&\begin{pmatrix}
C_{0}^{+} \\
C_{2}^{-}e^{iw_{s}t}
\end{pmatrix}
e^{-iw_{z}/vx}e^{ik_{0}x}e^{-iw_{0}t}\\
=&\begin{pmatrix}
\cos(w_{r}d/v) \\
-i\sin(w_{r}d/v)e^{iw_{s}t}
\end{pmatrix}
e^{-iw_{z}/vx}e^{ik_{0}x}e^{-iw_{0}t}\\
=&\begin{pmatrix}
\cos(w_{r}d/v)e^{i(k_{0}-w_{z}/v)x}e^{-iw_{0}t} \\
-i\sin(w_{r}d/v)e^{i(k_{0}-w_{z}/v)x}e^{-i(w_{0}-w_{s})t}
\end{pmatrix}
\end{align}
$領域ⅢとⅣの境界x=l_{1}における波動関数は$
\begin{align}
{\psi}_{Ⅲ}(x=l_{1},t)=
\begin{pmatrix}
\cos(w_{r}d/v)e^{i(k_{0}-w_{z}/v)l_{1}}e^{-iw_{0}t} \\
-i\sin(w_{r}d/v)e^{i(k_{0}-w_{z}/v)l_{1}}e^{-i(w_{0}-w_{s})t}
\end{pmatrix}
\end{align}
$領域Ⅳにおける磁場をB=\frac{{\hbar}w}{\mu}とすると、この領域ではスピン上成分が波数$
\begin{align}
k_{0}-w/v
\end{align}
スピン下成分が波数
\begin{align}
k_{0}-(w-w_{s})/v
\end{align}
で進んでいくので、領域Ⅳにおける波動関数は、
\begin{align}
{\psi}_{Ⅳ}(x,t)=
\begin{pmatrix}
\cos(w_{r}d/v)e^{i(k_{0}-w_{z}/v)l_{1}}e^{i(k_{0}-w/v)(x-l_{1})}e^{-iw_{0}t} \\
-i\sin(w_{r}d/v)e^{i(k_{0}-w_{z}/v)l_{1}}e^{i(k_{0}-(w-w_{s})/v)(x-l_{1})}e^{-i(w_{0}-w_{s})t}
\end{pmatrix}
\end{align}
領域Ⅴでは、両成分とも波数
\begin{align}
k_{0}-w_{z}/v
\end{align}
で進んでいくので、領域Ⅴにおける波動関数は
\begin{align}
{\psi}_{Ⅴ}(x,t)=
\begin{pmatrix}
\cos(w_{r}d/v)e^{i(k_{0}-w_{z}/v)l_{1}}e^{i(k_{0}-w/v)(l_{2}-l_{1})}e^{i(k_{0}-w_{z}/v)(x-l_{2})}e^{-iw_{0}t} \\
-i\sin(w_{r}d/v)e^{i(k_{0}-w_{z}/v)l_{1}}e^{i(k_{0}-(w-w_{s})/v)(l_{2}-l_{1})}e^{i(k_{0}-w_{z}/v)(x-l_{2})}e^{-i(w_{0}-w_{s})t}
\end{pmatrix}
\end{align}
$領域ⅤとⅥの境界x=l_{3}における波動関数は$
\begin{align}
{\psi}_{Ⅴ}(x=l_{3},t)=
\begin{pmatrix}
{\alpha}e^{-iw_{0}t} \\
{\beta}e^{-i(w_{0}-w_{s})t}
\end{pmatrix}
\end{align}
ここで、
\begin{align}
{\alpha}=\cos\left(w_{r}d/v\right)e^{i(k_{0}-w_{z}/v)l_{1}}e^{i\left(k_{0}-w/v\right)(l_{2}-l_{1})}e^{i\left(k_{0}-w_{z}/v\right)(l_{3}-l_{2})}
\end{align}
\begin{align}
{\beta}=i\sin\left(w_{r}d/v\right)e^{i(k_{0}-w_{z}/v)l_{1}}e^{i\left(k_{0}-(w-w_{s})/v\right)(l_{2}-l_{1})}e^{i\left(k_{0}-w_{z}/v\right)(l_{3}-l_{2})}
\end{align}
$領域Ⅵにおける波動関数のうち、エネルギーがE_{n}であるものは$
\begin{align}
{\psi}_{Ⅵ}(x,t)=
\begin{pmatrix}
(D_{n}^{+}e^{ik_{n}^{+}x}+D_{n}^{-}e^{ik_{n}^{-}x} )e^{-i(w_{n}+w_{s}/2)t}\\
(D_{n}^{+}e^{ik_{n}^{+}x}-D_{n}^{-}e^{ik_{n}^{-}x} )e^{-i(w_{n}-w_{s}/2)t}
\end{pmatrix}
\end{align}
である。ここで、
\begin{align}
\frac{{{\hbar}^2}{k_{n}^{\pm}}^2}{2m}{\pm}{\hbar}w_{r}={\hbar}w_{n}
\end{align}
$が成り立つ。x=l_{3}における接続により$
\begin{align}
{\alpha}e^{-iw_{0}t}=\left(D_{n}^{+}e^{ik_{n}^{+}l_{3}}+D_{n}^{-}e^{ik_{n}^{-}l_{3}}\right)e^{-i\left(w_{n}+w_{s}/2\right)t}
\end{align}
\begin{align}
{\beta}e^{-i(w_{0}-w_{s})t}=\left(D_{n}^{+}e^{ik_{n}^{+}l_{3}}-D_{n}^{-}e^{ik_{n}^{-}l_{3}}\right)e^{-i\left(w_{n}-w_{s}/2\right)t}
\end{align}
すると
\begin{align}
w_{2}=w_{0}-w_{s}/2
\end{align}
$として、n=2のもの以外は0になる。$
\begin{align}
{\alpha}=\left(D_{2}^{+}e^{ik_{2}^{+}l_{3}}+D_{2}^{-}e^{ik_{2}^{-}l_{3}}\right)
\end{align}
\begin{align}
{\beta}=\left(D_{2}^{+}e^{ik_{2}^{+}l_{3}}-D_{2}^{-}e^{ik_{2}^{-}l_{3}}\right)
\end{align}
よって
\begin{align}
D_{2}^{+}=\frac{1}{2}e^{-ik_{2}^{+}l_{3}}({\alpha}+{\beta})
\end{align}
\begin{align}
D_{2}^{-}=\frac{1}{2}e^{-ik_{2}^{-}l_{3}}({\alpha}-{\beta})
\end{align}
ゆえに、領域Ⅵにおける波動関数は
\begin{align}
{\psi}_{Ⅵ}(x,t)=\frac{1}{2}
\begin{pmatrix}
\left(({\alpha}+{\beta})e^{ik_{2}^{+}(x-l_{3})}+({\alpha}-{\beta})e^{ik_{2}^{-}(x-l_{3})}\right)e^{-iw_{0}t}\\
\left(({\alpha}+{\beta})e^{ik_{2}^{+}(x-l_{3})}-({\alpha}-{\beta})e^{ik_{2}^{-}(x-l_{3})}\right)e^{-i(w_{0}-w_{s})t}
\end{pmatrix}
\end{align}
領域Ⅶでは両成分とも波数
\begin{align}
k_{0}-w_{z}/v
\end{align}
で進んでいくので、波動関数は
\begin{align}
{\psi}_{Ⅶ}(x,t)=\frac{1}{2}
\begin{pmatrix}
\left(({\alpha}+{\beta})e^{ik_{2}^{+}(l_{4}-l_{3})}+({\alpha}-{\beta})e^{ik_{2}^{-}(l_{4}-l_{3})}\right)e^{i(k_{0}-w_{z}/v)(x-l_{4})}e^{-iw_{0}t}\\
\left(({\alpha}+{\beta})e^{ik_{2}^{+}(l_{4}-l_{3})}-({\alpha}-{\beta})e^{ik_{2}^{-}(l_{4}-l_{3})}\right)e^{i(k_{0}-w_{z}/v)(x-l_{4})}e^{-i(w_{0}-w_{s})t}
\end{pmatrix}
\end{align}
スピン上成分の絶対値の2乗は
\begin{align}
\left|({\psi}_{Ⅶ}(x,t))_{+}\right|^2=\frac{1}{4}\left|({\alpha}+{\beta})e^{ik_{2}^{+}(l_{4}-l_{3})}+({\alpha}-{\beta})e^{ik_{2}^{-}(l_{4}-l_{3})}\right|^2
\end{align}
ここで、
\begin{align}
\frac{{{\hbar}^2}{k_{n}^{\pm}}^2}{2m}{\pm}{\hbar}w_{r}={\hbar}w_{n}
\end{align}
により、
\begin{align}
\frac{{{\hbar}^2}{k_{2}^{\pm}}^2}{2m}{\pm}{\hbar}w_{r}={\hbar}w_{2}={\hbar}(w_{0}-w_{s}/2)={\hbar}(w_{0}-w_{z})
\end{align}
\begin{align}
k_{2}^{\pm}{\simeq}k_{0}-\frac{w_{z}{\pm}w_{r}}{v}
\end{align}
よって
\begin{align}
|({\psi}_{Ⅶ}(x,t))_{+}|^2=&\frac{1}{4}\left|({\alpha}+{\beta})e^{-iw_{r}/v(l_{4}-l_{3})}+({\alpha}-{\beta})e^{iw_{r}/v(l_{4}-l_{3})}\right|^2\\
=&\left|{\alpha}\cos\left(\frac{w_{r}}{v}(l_{4}-l_{3})\right)-i{\beta}\sin\left(\frac{w_{r}}{v}(l_{4}-l_{3})\right)\right|^2\\
=&\left|e^{i(k_{0}-w_{z}/v)(l_{2}-l_{1})}\cos\left(\frac{w_{r}}{v}d\right)\cos\left(\frac{w_{r}}{v}(l_{4}-l_{3})\right)-e^{i(k_{0}+(w-w_{s})/v)(l_{2}-l_{1})}\sin\left(\frac{w_{r}}{v}d\right)\sin\left(\frac{w_{r}}{v}(l_{4}-l_{3})\right)\right|^2
\end{align}







