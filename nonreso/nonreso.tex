\section{必ずしも共鳴条件を満たさない場合}
これまではシンプルにスピン干渉という現象を見るために、共鳴条件が満たされる場合のみを扱ってきた。しかし、実際の解析では共鳴条件を満たさない場合も扱う必要があるため、この章ではより一般的に必ずしも共鳴条件を満たさない場合に何が起こるかを見てゆく。


\subsection{スピンフリッパーの機能}
スピン干渉について考える前に、共鳴条件を満たさない場合のスピンフリッパーの機能について考えよう。

\paragraph{舞台とあらすじ}
一様磁場中に理想化されたRFスピンフリッパーがひとつ置かれた状況を考える。系は3つの領域I,II,IIIからなり、全体に$z$方向一様磁場$B_z$が、領域IIに$x$方向振動磁場$2B_r \cos \omega_s t$がかけられている。
\begin{figure}[h]
\centering
\includegraphics[height=3cm]{nonreso/whatwhyhow/Resonance_what_setting.pdf}
\end{figure}

これから述べるように、スピン上向きの中性子が領域Iから速度$v$で入射し領域IIを通って領域IIIに抜けるとき、領域IIIにおけるスピン上向き中性子の存在確率は
\begin{equation}
|\psi_{\mathrm{III}}^+|^2=\cos^2 \frac{\omega_A}{v}d+\left(\frac{\epsilon}{\omega_A}\right)^2\sin^2\frac{\omega_A}{v}d \label{Resonance_flip}
\end{equation}
となる。ここで$\epsilon=\omega_s/2-\omega_z,\omega_A=\sqrt{\epsilon^2+\omega_r^2}$であり、$\omega_z=|\mu_n|B_z,\omega_r=|\mu_n|B_r$、$\mu_n$は中性子の磁気モーメント、$d$は領域IIの幅である。ここで定義した
\begin{equation}
\epsilon=\frac{\omega_s}{2}-\omega_z
\end{equation}
という量は共鳴からのずれを表す指標であり、この章で非常に重要な役割を演じる。

式(\ref{Resonance_flip})を見ると$\epsilon=0$がなりたてば、$\omega_rd/v=\pi/2$を満たす速度の中性子に対して、領域IIIにおけるスピン上向き粒子の存在確率が0、すなわちスピン反転率が1になることがわかる。逆に$\epsilon \neq 0$のときはどのような速度の中性子に対しても反転率は1となり得ない。このように、$\epsilon=0$を満たす周波数の振動磁場をかけたときに限りスピンフリッパーを通り抜けた中性子のスピンが完全に反転し得る。これをスピンフリッパーの共鳴と呼び、$\epsilon=0$を共鳴条件と呼んだのであった。

\renewcommand{\arraystretch}{1.5}

\paragraph{入射波}
スピンの量子化軸を$z$軸に選ぶと、領域IにおけるShr$\ddot{\mathrm{o}}$dinger方程式は%なぜか\''{o}が使えない
\begin{equation}
i\frac{\del \psi_\mathrm{I}}{\del t}= \Biggl[-\frac{1}{2m} \frac{\del^2}{\del x^2} \underbrace{-\mu_nB_z}_{+|\mu_n|Bz =\omega_z} \sigma_z\Biggr] \psi_\mathrm{I}
\end{equation}
入射中性子のスピンは上向きとしたので、そのエネルギーを$\omega_0$とすると、領域Iにおける波動関数の入射成分はスピン上下の2成分表示で
\begin{equation}
\psi_\mathrm{I}^{\mathrm{in}}=\begin{pmatrix} 1\\0 \end{pmatrix} \e^{i k_0^+ x -i\omega_0 t}
\end{equation}
と書ける。ここで$k_0^+=\sqrt{2m(\omega_0 -\omega_z)}$。
\renewcommand{\arraystretch}{1}

\paragraph{領域II}
領域IIにおける磁場を
\begin{equation}
\bm{B}_{\mathrm{II}}=\hat{\bm{x}} (2B_r)\cos \omega_s t+\hat{\bm{z}} B_z
\end{equation}
と書く。ただし$\hat{\bm{x}},\hat{\bm{z}}$はそれぞれx,z軸正の向きの単位ベクトル。$|B_r| \ll |B_z|$のときには、\ref{pi2flipper_sec}章で述べた理由により振動磁場を次のような$xy$平面上の回転磁場に置き換えてよい:
\begin{equation}
\hat{\bm{x}} (2B_r)\cos \omega_s t \rightarrow \hat{\bm{x}} B_r \cos \omega_s t+\hat{\bm{y}} \sin \omega_s t
\end{equation}
したがって$|B_r| \ll |B_z|$のとき領域IIにおける磁場は
\begin{equation}
\bm{B}_{\mathrm{II}}\simeq \hat{\bm{x}} B_r\cos \omega_s t+\hat{\bm{y}} B_r \sin \omega_s t+\hat{\bm{z}} B_z
\end{equation}
となる。このとき領域IIにおけるShr$\ddot{\mathrm{o}}$dinger方程式は
\begin{align}
i\frac{\del \psi_{\mathrm{II}}}{\del t} &=\left[-\frac{1}{2m} \frac{\del^2}{\del x^2} -\bm{\mu}_n\cdot\bm{B}_\mathrm{II}\right] \psi_\mathrm{II}\notag \\
&=\left[-\frac{1}{2m} \frac{\del^2}{\del x^2} + |\mu_n| (\sigma_x B_r \cos\omega_s t+\sigma_y B_r\sin\omega_s t+\sigma_zB_z)\right] \psi_\mathrm{II} \notag \\
&=\left[-\frac{1}{2m} \frac{\del^2}{\del x^2} +\begin{pmatrix} \omega_z &\omega_r \e^{-i\omega_s t} \\\omega_r \e^{i\omega_s t} &-\omega_z \end{pmatrix}\right] \psi_\mathrm{II}
\end{align}
となる。まずハミルトニアンの時間依存性を除くために、$z$軸まわりの角速度$-\omega_s$の回転を表すユニタリ変換
\begin{equation}
U_T=\exp\left[+iS_z\omega_s t\right] =\exp\left[i\sigma_z\frac{\omega_s t}{2}\right]=\begin{pmatrix} \e^{i\omega_st/2} &0\\0&\e^{-i\omega_st/2}\end{pmatrix}
\end{equation}
を用いて
\begin{equation}
U_T \begin{pmatrix} \omega_z &\omega_r \e^{-i\omega_s t} \\\omega_r \e^{i\omega_s t} &-\omega_z \end{pmatrix} U_T^\dagger =\begin{pmatrix} \omega_z &\omega_r\\ \omega_r&-\omega_z \end{pmatrix}
\end{equation}
がなりたつことに注意すると、
\begin{equation}
i\frac{\del}{\del t}(U_T \psi_\mathrm{II}) =\left[-\frac{1}{2m} \frac{\del^2}{\del x^2} +\begin{pmatrix} -(\frac{\omega_s}{2}-\omega_z) &\omega_r\\ \omega_r &\frac{\omega_s}{2}-\omega_z \end{pmatrix}\right] (U_T \psi_\mathrm{II})
\end{equation}
を得る。さらにこのハミルトニアンを対角的にするために
\begin{equation}
\begin{pmatrix} -(\frac{\omega_s}{2}-\omega_z) &\omega_r\\ \omega_r &\frac{\omega_s}{2}-\omega_z \end{pmatrix}=\omega_A\begin{pmatrix} \cos \theta&\sin \theta\\ \sin \theta &-\cos \theta \end{pmatrix}
\end{equation}
と書く。ここで$\epsilon=\omega_s/2-\omega_z,\omega_A=\sqrt{\epsilon^2+\omega_r^2}$として
\begin{equation}
\cos \theta=\frac{-\epsilon}{\omega_A} , \quad \sin \theta=\frac{\omega_r}{\omega_A}
\end{equation}
である。$y$軸まわりの角度$-\theta$回転を表すユニタリ変換
\begin{equation}
U_D=\exp\left[+iS_y \theta\right] =\exp\left[i\sigma_y \frac{\theta}{2}\right] =\begin{pmatrix} \cos\frac{\theta}{2}& \sin\frac{\theta}{2}\\  -\sin\frac{\theta}{2}& \cos\frac{\theta}{2}\end{pmatrix}
\end{equation}
を用いて
\begin{equation}
U_D \begin{pmatrix} \cos \theta&\sin \theta\\ \sin \theta &-\cos \theta \end{pmatrix} U_D^\dagger =\begin{pmatrix} 1&0\\0&-1\end{pmatrix}
\end{equation}
がなりたつことに注意すると、
\begin{equation}
i\frac{\del}{\del t}(U_DU_T \psi_\mathrm{II}) =\left[-\frac{1}{2m} \frac{\del^2}{\del x^2} +\omega_A \sigma_Z \right] (U_DU_T \psi_\mathrm{II})
\end{equation}
を得る。

\renewcommand{\arraystretch}{1.5}
したがって、領域IIにおける中性子のエネルギーを$\omega_n$とすると、定数$A_n^\pm,B_n^\pm$を用いて
\begin{equation}
\psi_D \equiv U_D U_T \psi_\mathrm{II} =\begin{pmatrix} A_n^+ \e^{ik_n^+x}+B_n^+\e^{-ik_n^+x} \\ A_n^- \e^{ik_n^-x}+B_n^-\e^{-ik_n^-x} \end{pmatrix} \e^{-i\omega_n t}
\end{equation}
と書けるから、領域IIにおける波動関数は
\begin{align}
\psi_\mathrm{II}&=U_T^\dagger U_D^\dagger \psi_D \notag \\
&=\begin{pmatrix} \e^{-i\omega_st/2} &0\\0&\e^{+i\omega_st/2}\end{pmatrix} \begin{pmatrix} \cos\frac{\theta}{2}& -\sin\frac{\theta}{2}\\  \sin\frac{\theta}{2}& \cos\frac{\theta}{2}\end{pmatrix} \begin{pmatrix} A_n^+ \e^{ik_n^+x}+B_n^+\e^{-ik_n^+x} \\ A_n^- \e^{ik_n^-x}+B_n^-\e^{-ik_n^-x} \end{pmatrix} \e^{-i\omega_n t} \notag \\
&=\begin{pmatrix} \left(A_n^+ \cos \frac{\theta}{2} \e^{ik_n^+x}+ B_n^+ \cos \frac{\theta}{2} \e^{-ik_n^+x} - A_n^- \sin \frac{\theta}{2} \e^{ik_n^-x}- B_n^- \sin \frac{\theta}{2} \e^{-ik_n^-x}\right) \e^{-i(\omega_n+\omega_s/2)t} \\ \left(A_n^+ \sin \frac{\theta}{2} \e^{ik_n^+x}+ B_n^+ \sin \frac{\theta}{2} \e^{-ik_n^+x} + A_n^- \cos \frac{\theta}{2} \e^{ik_n^-x}+ B_n^- \cos \frac{\theta}{2} \e^{-ik_n^-x}\right) \e^{-i(\omega_n-\omega_s/2)t} \end{pmatrix} \label{Resonance_psi2}
\end{align}
と書ける。ここで$k_n^\pm=\sqrt{2m(\omega_n \mp \omega_A)}$。

\paragraph{領域IとIIの接続}
入射波$\psi_\mathrm{I}^\mathrm{in}$と$\psi_\mathrm{II}$が任意の時刻$t$で領域IとIIの境界($x=0$)において接続するためには
\begin{equation}
\omega_n=\omega_1 \equiv \omega_0 -\frac{\omega_s}{2}
\end{equation}
が必要である。そのとき
\begin{equation}
\psi_\mathrm{II}=\begin{pmatrix} \left(A_1^+ \cos \frac{\theta}{2} \e^{ik_1^+x}+ B_1^+ \cos \frac{\theta}{2} \e^{-ik_1^+x} - A_1^- \sin \frac{\theta}{2} \e^{ik_1^-x}- B_1^- \sin \frac{\theta}{2} \e^{-ik_1^-x}\right) \e^{-i\omega_0t} \\ \left(A_1^+ \sin \frac{\theta}{2} \e^{ik_1^+x}+ B_1^+ \sin \frac{\theta}{2} \e^{-ik_1^+x} + A_1^- \cos \frac{\theta}{2} \e^{ik_1^-x}+ B_1^- \cos \frac{\theta}{2} \e^{-ik_1^-x}\right) \e^{-i(\omega_0-\omega_s)t} \end{pmatrix}
\end{equation}
ここで$k_1^\pm=\sqrt{2m(\omega_0-\omega_s/2 \mp \omega_A)}$。

またこのことから、領域Iでの反射波を含めた全波動関数$\psi_\mathrm{I}$は
\begin{equation}
\psi_\mathrm{I} =I_0^+ \psi_\mathrm{I}^\mathrm{in} +\begin{pmatrix} R_0^+ \e^{-ik_0^+x}\e^{ -i\omega_0 t} \\ R_2^- \e^{-ik_2^-x}\e^{ -i (\omega_0-\omega_s)t} \end{pmatrix} =\begin{pmatrix} (I_0^+ \e^{ik_0^+x} +R_0^+\e^{-ik_0^+x} )\e^{-i\omega_0 t} \\ R_2^- \e^{-ik_2^-x}\e^{ -i (\omega_0-\omega_s)t} \end{pmatrix}
\end{equation}
と書ける。ここで$I_0^+,R_0^+,R_2^-$は定数であり、$k_0^+ =\sqrt{2m(\omega_0 -\omega_z)},k_2^- =\sqrt{2m(\omega_0 -\omega_s +\omega_z)}$である。

さて、ここで領域IとIIの接続を考えると、次の4つの式がなりたつ:
\begin{align}
\left\{\begin{array}{l}
A_1^+ \cos \frac{\theta}{2} +B_1^+\cos \frac{\theta}{2} -A_1^-\sin\frac{\theta}{2} -B_1^-\sin\frac{\theta}{2} = I_0^+ +R_0^+ \\
k_1^+(A_1^+ \cos \frac{\theta}{2} -B_1^+\cos \frac{\theta}{2} )-k_1^-(A_1^-\sin\frac{\theta}{2} -B_1^-\sin\frac{\theta}{2} )=k_0^+(I_0^--R_0^+)\\
A_1^+ \sin \frac{\theta}{2} +B_1^+\sin \frac{\theta}{2} +A_1^-\cos\frac{\theta}{2} +B_1^-\cos\frac{\theta}{2} = R_2^- \\
k_1^+(A_1^+ \sin \frac{\theta}{2} -B_1^+\sin \frac{\theta}{2}) +k_1^-(A_1^-\cos\frac{\theta}{2} -B_1^-\cos\frac{\theta}{2}) = -k_2^-R_2^-
\end{array} \right. \label{Resonance_setsuzoku}
\end{align}

\paragraph{近似}
いま中性子の入射エネルギー$\omega_0$は、$\omega_z$や$\omega_s,\omega_r$などと比べて十分大きいとする。このとき
\begin{align}
&k_0^+ =\sqrt{2m(\omega_0 -\omega_z)} \simeq k_0 -\frac{\omega_z}{v} \label{Resonance_k0+} \\
&k_1^{\pm} = \sqrt{2m(\omega_0-\omega_s/2 \mp \omega_A)} \simeq k_0 -\frac{\omega_s}{2v} \mp \frac{\omega_A}{v} \label{Resonance_k1+-} \\
&k_2^- =\sqrt{2m(\omega_0 -\omega_s +\omega_z)} \simeq k_0 -\frac{\omega_s}{v} +\frac{\omega_z}{v}\label{Resonance_k2-}
\end{align}
となる。ここで$k_0 \equiv \sqrt{2m\omega_0}$とした。さらに式(\ref{Resonance_setsuzoku})のように指数の肩以外に$k_n^\pm$が降りているときは
\begin{equation}
k_n^\pm \simeq k_0\label{Resonance_kn+-}
\end{equation}
と近似する。後で見るようにこれは反射波を無視することと同値である。

この近似の下で式(\ref{Resonance_setsuzoku})は次のように変形される:
\begin{align}
&\left\{\begin{array}{l}
A_1^+ \cos \frac{\theta}{2} +B_1^+\cos \frac{\theta}{2} -A_1^-\sin\frac{\theta}{2} -B_1^-\sin\frac{\theta}{2} = I_0^+ +R_0^+ \\
k_0(A_1^+ \cos \frac{\theta}{2} -B_1^+\cos \frac{\theta}{2} )-k_0(A_1^-\sin\frac{\theta}{2} -B_1^-\sin\frac{\theta}{2} )=k_0(I_0^--R_0^+)\\
A_1^+ \sin \frac{\theta}{2} +B_1^+\sin \frac{\theta}{2} +A_1^-\cos\frac{\theta}{2} +B_1^-\cos\frac{\theta}{2} = R_2^- \\
k_0(A_1^+ \sin \frac{\theta}{2} -B_1^+\sin \frac{\theta}{2}) +k_0(A_1^-\cos\frac{\theta}{2} -B_1^-\cos\frac{\theta}{2}) = -k_0R_2^-
\end{array} \right.  \notag \\
&\Rightarrow \left\{\begin{array}{l}
A_1^+ \cos \frac{\theta}{2} +B_1^+\cos \frac{\theta}{2} -A_1^-\sin\frac{\theta}{2} -B_1^-\sin\frac{\theta}{2} = I_0^+ +R_0^+ \\
A_1^+ \cos \frac{\theta}{2} -B_1^+\cos \frac{\theta}{2} -A_1^-\sin\frac{\theta}{2} -B_1^-\sin\frac{\theta}{2} =I_0^--R_0^+\\
A_1^+ \sin \frac{\theta}{2} +B_1^+\sin \frac{\theta}{2} +A_1^-\cos\frac{\theta}{2} +B_1^-\cos\frac{\theta}{2} = R_2^- \\
A_1^+ \sin \frac{\theta}{2} -B_1^+\sin \frac{\theta}{2} +A_1^-\cos\frac{\theta}{2} -B_1^-\cos\frac{\theta}{2} = -R_2^-
\end{array} \right.  \notag \\
&\Rightarrow \left\{\begin{array}{l}
A_1^+ \cos \frac{\theta}{2} -A_1^-\sin\frac{\theta}{2}= I_0^+ \\
B_1^+\cos \frac{\theta}{2} -B_1^-\sin\frac{\theta}{2} =R_0^+\\
A_1^+ \sin \frac{\theta}{2}+A_1^-\cos\frac{\theta}{2}= 0 \\
B_1^+\sin \frac{\theta}{2} -B_1^-\cos\frac{\theta}{2} = R_2^-
\end{array} \right.  \notag \\
&\Rightarrow \left\{\begin{array}{l}
A_1^+ = I_0^+\cos\frac{\theta}{2} \\
A_1^- =-I_0^+\sin\frac{\theta}{2} \\
B_1^+=R_0^+\cos\frac{\theta}{2}+R_2^-\sin\frac{\theta}{2} \\
B_1^-=R_2^-\cos\frac{\theta}{2}-R_0\sin\frac{\theta}{2}
\end{array} \right. \label{Resonance_1and2}
\end{align}

\paragraph{領域III}
領域IIIにおけるShr$\ddot{\mathrm{o}}$dinger方程式は領域Iと同じで
\begin{equation}
i\frac{\del \psi_\mathrm{III}}{\del t}= \left[-\frac{1}{2m} \frac{\del^2}{\del x^2} +\omega_z \sigma_z\right] \psi_\mathrm{III}
\end{equation}
である。$\psi_\mathrm{II}$と$\psi_\mathrm{III}$が任意の時刻$t$で領域IIとIIIの境界($x=d$)において接続するためには、式(\ref{Resonance_psi2})より、領域IIIにおける中性子のエネルギーがスピン上向きに対しては$\omega_0$、スピン下向きに対しては$\omega_0-\omega_s$であることが必要である。よって領域IIIにおける波動関数$\psi_\mathrm{III}$は定数$C_0^+,C_2^-$を用いて
\begin{equation}
\psi_\mathrm{III}=\begin{pmatrix} C_0^+ \e^{ik_0^+ x}\e^{-i\omega_0 t} \\ C_2^- \e^{ik_2^- x}\e^{-i(\omega_0-\omega_s) t} \end{pmatrix}
\end{equation}
と書ける。ここで$k_0^+=\sqrt{2m(\omega_0-\omega_z)},k_2^-=\sqrt{2m(\omega_0-\omega_s+\omega_z)}$。

\paragraph{領域IIとIIIの接続}
次に領域IIとIIIの接続を考える。前述の通り、波数の内、指数の肩に乗っているものについては式(\ref{Resonance_k0+}),(\ref{Resonance_k1+-}),(\ref{Resonance_k2-})のように、指数の肩から降りているものについては式(\ref{Resonance_kn+-})のように近似する。すると次の4つの式がなりたつ:
\begin{align}
&\left\{\begin{array}{l}
\left(A_1^+ \cos \frac{\theta}{2} \e^{-i\frac{\omega_A}{v}d}+B_1^+\cos \frac{\theta}{2} \e^{i\frac{\omega_A}{v}d}-A_1^-\sin\frac{\theta}{2}\e^{i\frac{\omega_A}{v}d} -B_1^-\sin\frac{\theta}{2}\e^{-i\frac{\omega_A}{v}d} \right)\e^{i(k_0-\frac{\omega_s}{2v})d}= C_0^+\e^{i(k_0-\frac{\omega_z}{v})d} \\
\left(A_1^+ \cos \frac{\theta}{2} \e^{-i\frac{\omega_A}{v}d}-B_1^+\cos \frac{\theta}{2} \e^{i\frac{\omega_A}{v}d}-A_1^-\sin\frac{\theta}{2}\e^{i\frac{\omega_A}{v}d} +B_1^-\sin\frac{\theta}{2}\e^{-i\frac{\omega_A}{v}d} \right)\e^{i(k_0-\frac{\omega_s}{2v})d}= C_0^+\e^{i(k_0-\frac{\omega_z}{v})d} \\
\left(A_1^+ \sin \frac{\theta}{2} \e^{-i\frac{\omega_A}{v}d}+B_1^+\sin \frac{\theta}{2} \e^{i\frac{\omega_A}{v}d}+A_1^-\cos\frac{\theta}{2}\e^{i\frac{\omega_A}{v}d} +B_1^-\cos\frac{\theta}{2}\e^{-i\frac{\omega_A}{v}d} \right)\e^{i(k_0-\frac{\omega_s}{2v})d}= C_2^-\e^{i(k_0-\frac{\omega_s}{v}+\frac{\omega_z}{v})d} \\
\left(A_1^+ \sin \frac{\theta}{2} \e^{-i\frac{\omega_A}{v}d}-B_1^+\sin \frac{\theta}{2} \e^{i\frac{\omega_A}{v}d}+A_1^-\cos\frac{\theta}{2}\e^{i\frac{\omega_A}{v}d} -B_1^-\cos\frac{\theta}{2}\e^{-i\frac{\omega_A}{v}d} \right)\e^{i(k_0-\frac{\omega_s}{2v})d}= C_2^-\e^{i(k_0-\frac{\omega_s}{v}+\frac{\omega_z}{v})d} \\
\end{array} \right.  \notag \\
&\Rightarrow \left\{\begin{array}{l}
\left(A_1^+ \cos \frac{\theta}{2} \e^{-i\frac{\omega_A}{v}d}+B_1^+\cos \frac{\theta}{2} \e^{i\frac{\omega_A}{v}d}-A_1^-\sin\frac{\theta}{2}\e^{i\frac{\omega_A}{v}d} -B_1^-\sin\frac{\theta}{2}\e^{-i\frac{\omega_A}{v}d} \right)\e^{-i\frac{\frac{\omega_s}{2}-\omega_z}{v}d}= C_0^+\\
\left(A_1^+ \cos \frac{\theta}{2} \e^{-i\frac{\omega_A}{v}d}-B_1^+\cos \frac{\theta}{2} \e^{i\frac{\omega_A}{v}d}-A_1^-\sin\frac{\theta}{2}\e^{i\frac{\omega_A}{v}d} +B_1^-\sin\frac{\theta}{2}\e^{-i\frac{\omega_A}{v}d} \right)\e^{-i\frac{\frac{\omega_s}{2}-\omega_z}{v}d}= C_0^+\\
\left(A_1^+ \sin \frac{\theta}{2} \e^{-i\frac{\omega_A}{v}d}+B_1^+\sin \frac{\theta}{2} \e^{i\frac{\omega_A}{v}d}+A_1^-\cos\frac{\theta}{2}\e^{i\frac{\omega_A}{v}d} +B_1^-\cos\frac{\theta}{2}\e^{-i\frac{\omega_A}{v}d} \right)\e^{i\frac{\frac{\omega_s}{2}-\omega_z}{v}d}= C_2^-\\
\left(A_1^+ \sin \frac{\theta}{2} \e^{-i\frac{\omega_A}{v}d}-B_1^+\sin \frac{\theta}{2} \e^{i\frac{\omega_A}{v}d}+A_1^-\cos\frac{\theta}{2}\e^{i\frac{\omega_A}{v}d} -B_1^-\cos\frac{\theta}{2}\e^{-i\frac{\omega_A}{v}d} \right)\e^{i\frac{\frac{\omega_s}{2}-\omega_z}{v}d}= C_2^- \\
\end{array} \right.  \notag \\
&\Rightarrow \left\{\begin{array}{l}
\left(A_1^+ \cos \frac{\theta}{2} \e^{-i\frac{\omega_A}{v}d}-A_1^-\sin\frac{\theta}{2}\e^{i\frac{\omega_A}{v}d} \right)\e^{-i\frac{\epsilon}{v} d}= C_0^+\\
\left(B_1^+\cos \frac{\theta}{2} \e^{i\frac{\omega_A}{v}d}-B_1^-\sin\frac{\theta}{2}\e^{-i\frac{\omega_A}{v}d} \right)\e^{-i\frac{\epsilon}{v} d}= 0\\
\left(A_1^+ \sin \frac{\theta}{2} \e^{-i\frac{\omega_A}{v}d}+A_1^-\cos\frac{\theta}{2}\e^{i\frac{\omega_A}{v}d} \right)\e^{i\frac{\epsilon}{v} d}= C_2^- \\
\left(B_1^+\sin \frac{\theta}{2} \e^{i\frac{\omega_A}{v}d}+B_1^-\cos\frac{\theta}{2}\e^{-i\frac{\omega_A}{v}d} \right)\e^{i\frac{\epsilon}{v} d}= 0
\end{array} \right. \notag \\
&\Rightarrow \left\{\begin{array}{l}
C_0^+=\left(A_1^+ \cos \frac{\theta}{2} \e^{-i\frac{\omega_A}{v}d}-A_1^-\sin\frac{\theta}{2}\e^{i\frac{\omega_A}{v}d} \right)\e^{-i\frac{\epsilon}{v} d}\\
C_2^-=\left(A_1^+ \sin \frac{\theta}{2} \e^{-i\frac{\omega_A}{v}d}+A_1^-\cos\frac{\theta}{2}\e^{i\frac{\omega_A}{v}d} \right)\e^{i\frac{\epsilon}{v} d}\\
B_1^+=0\\
B_1^-= 0
\end{array} \right. \label{Resonance_2and3}
\end{align}
%ここで$\epsilon=\omega_s/2-\omega_z$とした。
%\end{comment}


\paragraph{結末}
したがって式(\ref{Resonance_1and2}),(\ref{Resonance_2and3})より
\begin{equation}
\left\{ \begin{array}{l}
A_1^+=I_0^+ \cos \frac{\theta}{2} \\
A_1^-=-I_0^+ \cos \frac{\theta}{2} \\
B_1^+=0 \\
B_1^-=0 \\
C_0^+=I_0^+\left(\sin^2 \frac{\theta}{2} \e^{i\frac{\omega_A}{v}d} +\cos^2 \frac{\theta}{2} \e^{-i\frac{\omega_A}{v}d} \right)  \e^{-i\frac{\epsilon}{v}d} \\
C_2^-=I_0^+ \sin\frac{\theta}{2}\cos\frac{\theta}{2} \left(\e^{-i\frac{\omega_A}{v}d}-\e^{i\frac{\omega_A}{v}d}\right) \e^{i\frac{\epsilon}{v}d}\\
R_0^+=0\\
R_2^-=0
\end{array} \right.
\end{equation}
を得る。前述の通り$B_1^\pm=R_0^+=R_2^-=0$、すなわち全ての境界における反射波はゼロとなる。いま$|\psi_\mathrm{I}|^2=|I_0^+|^2=1$と規格化すると、$I_0^+=1$ととってよい。そのとき

%\renewcommand{\arraystretch}{1}
\begin{align}
\psi_\mathrm{I}(x,t)&=\begin{pmatrix} 1\\0\end{pmatrix} \e^{i (k_0-\frac{\omega_z}{v})x}\e^{-i\omega_0t}\\
\psi_\mathrm{II}(x,t)&=\begin{pmatrix} \left(\sin^2 \frac{\theta}{2} \e^{i\frac{\omega_A}{v}x} +\cos^2 \frac{\theta}{2} \e^{-i\frac{\omega_A}{v}x} \right) \e^{i(k_0-\frac{\omega_s}{v}) x} \e^{-i\omega_0 t} \\ \sin\frac{\theta}{2}\cos\frac{\theta}{2} \left(\e^{-i\frac{\omega_A}{v}x}-\e^{i\frac{\omega_A}{v}x}\right) \e^{i(k_0-\frac{\omega_s}{v}) x} \e^{-i(\omega_0-\omega_s) t}\end{pmatrix}\\
\psi_\mathrm{III}(x,t)&=\begin{pmatrix} \left(\sin^2 \frac{\theta}{2} \e^{i\frac{\omega_A}{v}d} +\cos^2 \frac{\theta}{2} \e^{-i\frac{\omega_A}{v}d} \right)\e^{-i\frac{\epsilon}{v}d} \e^{i(k_0-\frac{\omega_z}{v})x}\e^{-i\omega_0t} \\ \sin\frac{\theta}{2}\cos\frac{\theta}{2} \left(\e^{-i\frac{\omega_A}{v}d}-\e^{i\frac{\omega_A}{v}d}\right)  \e^{i\frac{\epsilon}{v}d} \e^{i(k_0-\frac{\omega_s}{v}+\frac{\omega_z}{v})x}\e^{-i(\omega_0-\omega_s)t} \end{pmatrix}
\end{align}
となる。

ここで$\sin^2 \frac{\theta}{2}=\frac{1}{2}(1-\cos \theta) =\frac{1}{2} (1+\frac{\epsilon}{\omega_A}),\cos^2\frac{\theta}{2}=\frac{1}{2} (1+\cos\theta) =\frac{1}{2} (1-\frac{\epsilon}{\omega_A})$より
\begin{align}
\sin^2 \frac{\theta}{2} \e^{i\frac{\omega_A}{v}d} +\cos^2 \frac{\theta}{2} \e^{-i\frac{\omega_A}{v}d} &= \frac{1}{2} \left(1+\frac{\epsilon}{\omega_A}\right) \left(\cos \frac{\omega_A}{v}d+i\sin \frac{\omega_A}{v}d \right)+\frac{1}{2} \left(1-\frac{\epsilon}{\omega_A}\right) \left(\cos \frac{\omega_A}{v}d-i\sin \frac{\omega_A}{v}d \right) \notag \\
&=\cos \frac{\omega_A}{v}d +i\frac{\epsilon}{\omega_A} \sin \frac{\omega_A}{v}d
\end{align}
また$\sin\theta=\omega_r/\omega_A$より
\begin{align}
\sin\frac{\theta}{2}\cos\frac{\theta}{2} \left(\e^{-i\frac{\omega_A}{v}d}-\e^{i\frac{\omega_A}{v}d}\right) &=-i \sin \theta \sin \frac{\omega_A}{v}d \notag \\
&= -i \frac{\omega_r}{\omega_A} \sin \frac{\omega_A}{v}d
\end{align}
ゆえに
\begin{align}
\psi_\mathrm{III}(x,t)=\begin{pmatrix} \left(\cos \frac{\omega_A}{v}d +i\frac{\epsilon}{\omega_A} \sin \frac{\omega_A}{v}d\right) \e^{-i\frac{\epsilon}{v} d} \e^{i(k_0-\frac{\omega_z}{v})x}\e^{-i\omega_0t} \\ -i \frac{\omega_r}{\omega_A} \sin \frac{\omega_A}{v}d  \, \e^{i\frac{\epsilon}{v}d} \e^{i(k_0-\frac{\omega_s}{v}+\frac{\omega_z}{v})x}\e^{-i(\omega_0-\omega_s)t} \end{pmatrix}
\end{align}
となる。したがって領域Iからスピン上向きの中性子を入射したとき、領域IIIでスピン上向きの中性子を観測する確率は
\begin{align}
\bigl|\psi_\mathrm{III}^+\bigr|^2 &=\biggl |\cos \frac{\omega_A}{v}d +i\frac{\epsilon}{\omega_A} \sin \frac{\omega_A}{v}d\biggr|^2 \notag \\
&=\cos^2 \frac{\omega_A}{v}d+\left(\frac{\epsilon}{\omega_A}\right)^2\sin^2\frac{\omega_A}{v}d
\end{align} \label{Resonance_1-reversal}
となる。

いま$\omega_r/\omega_z=0.5$とし、$\epsilon/\omega_z=0,0.3,0.5,1.0$の各場合についてスピンフリッパーを通過した中性子のスピン反転率($=1-|\psi_\mathrm{III}^+|^2$)と$\omega_r d/v$の関係を図\ref{Resonance_fig_reversalrate}に表す。
図\ref{Resonance_fig_reversalrate}より
\begin{equation}
\epsilon=\frac{\omega_s}{2}-\omega_z=0 \label{Resonance_resonance}
\end{equation}
がなりたてば、$\omega_r d/v =(2n+1)\pi/2 \ (n =0,1,2,\cdots)$を満たす速度の中性子に対して、領域IIIにおけるスピン上向き粒子の存在確率が0、すなわちスピン反転率が1になることがわかる。逆に$\epsilon \neq 0$のときはどのような速度の中性子に対しても反転率は1となり得ない。このように、式(\ref{Resonance_resonance})を満たす周波数の振動磁場をかけたときに限りスピンフリッパーを通り抜けた中性子のスピンが完全に反転し得る。これをスピンフリッパーの共鳴と呼び、式(\ref{Resonance_resonance})を共鳴条件と呼ぶ。
\begin{figure}[h]
\begin{center}
\includegraphics[width=10cm]{nonreso/whatwhyhow/resonance_reversalrate.pdf}
\caption{スピン反転率}
\label{Resonance_fig_reversal}
\end{center}
\end{figure}

%\begin{comment}
%\paragraph{振動磁場の反回転成分が無視できる理由}
%略(本の通り)
%\renewcommand{\arraystretch}{1}
%\end{comment}

%\clearpage

\subsection{スピン干渉の原理}